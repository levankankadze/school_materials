
\documentclass[a4paper]{article}

%% Language and font encodings
\usepackage[T1,T8K,T8M]{fontenc}
\usepackage[utf8]{inputenc}
\usepackage[english,georgian]{babel}

%% Sets page size and margins
\usepackage[a4paper,top=3cm,bottom=2cm,left=3cm,right=3cm,marginparwidth=1.75cm]{geometry}

%% Useful packages
\usepackage{amsmath}
\usepackage{graphicx}
\usepackage[colorinlistoftodos]{todonotes}
\usepackage[colorlinks=true, allcolors=blue]{hyperref}
\usepackage{float}
\usepackage{enumerate}
\usepackage{subfig}

\title{კეპლერის კანონები}
\author{ლეევან კანკაძე}

\begin{document}
	\maketitle
	
	\begin{abstract}
		კეპლერის კანონები.
	\end{abstract}
	
	\section{შენახვის კანონები}
	\textbf{01.} $m$ მასის უძრავ ბირთვს $V$ სიჩქარით ეჯახება $M$ მასის მოძრავი ბირთვი. იპოვეთ ბირთვების სიჩქარეები დაჯახების შემდეგ, თუ დაჯახება დრეკადია და ცენტრული.  
		
	\section{ჭოჭონაქები}
	\textbf{01.} იპოვეთ რა ძალით მოქმედებს ჭერზე, ნახატზე გამოსახული უმასო ჭოჭონაქების სისტემა. თოკები უჭიმვადია და უმასო,  თითოეული სხეულის მასაა $m$.  ხახუნი უგულებელყავით.
	 			\begin{figure}[H]
	 			\centering
           \includegraphics[width=0.2\columnwidth]{figures/03}
           \caption{A boat.}
           \label{fig:boat1}
        \end{figure}
	 
	\section{წრეწირზე მოძრაობა}
	 \textbf{01.} მოტოციკლეტისტი მოძრაობს ჰორიზონტალურ ზედაპირზე $v = 70$ კმ/სთ სიჩქარით, ბრუნდება $R = 100$ მ რადიუსის მოსახვევში, რა კუთხით უნდა გადაიხაროს რომ არ დაეცეს? \\
	 ამოხსნა\\
	 აქაც ხახუნის ძალაა, ძალა რომელცც აჩერებს მოტოცეკლისტს, $F_{fr} = \frac{m v^2}{R}$, საყრდენის რეაქციის ძალა $N = mg$. მომენტების წესი სიმძიმის ცენტრის მიმართ მომცემს განტოლებას $F_{fr}\cdot l \sin \alpha = N l \cos \alpha$. აქ მოცემული არაა $\mu$ და მაგიტომ გვჭირდება. ეს მომენტები.
	 
	\textbf{02.} რა მაქსიმალური $v$ სიჩქარით შეიძლება იმოძრავოს მანქანამ $\alpha$ კუთხით დახრილ სიბრტყეზე თუ სიმრუდის რადიუსია $R$ და ხახუნის კოეფიციენტი ბორბლებსა და გზას შორის არის $k$.
			\begin{figure}[H]
           \includegraphics[width=\linewidth = 0.6]{figures/02}
           \caption{A boat.}
           \label{fig:boat1}
        \end{figure}
	 
	\section{კეპლერის კანონები}
	\subsection{კეპლერის პირველი კანონი}
	პლანეტები მოძრაობს ელიფსებზე, რომელთა ერთ-ერთ ფოკუსში იმყოფება მზე.
	\subsection{კეპლერის მეორე კანონი}
	პლანეტის რადიუს-ვექტორი დროის ტოლ შუალედებში ტოლ ფართობებს მოხვეტს.
	\subsection{კეპლერის მესამე კანონი}
	პლანეტების გარშემოვლის პერიოდების კვადრატები ისე შეეფარდება ერთმანეთს, როგორც მათი ორბიტების დიდი ნახევარღერძების კუბები.
	\begin{equation}
		\frac{T_1^2}{T_2^2} = \frac{a_1^3}{a_2^3}
	\end{equation}
	\section{გრავიტაციული ურთიერთქმედების პოტენციალური ენერგია}
	$r$ მანძილით დაშორებული $m_1$ და $m_2$ მასის ნივთიერი წერტილების გრავიტაციული ურთიერთქმედების პოტენციალური ენერგიის ფორმულის მიღებას ინტეგრების ცოდნა სჭირდება. ჩვენ მოვიყვანთ შედეგს გამოყვანის გარეშე:
	\begin{equation}
		U = -G\frac{m_1 m_2}{r} + C
	\end{equation}
სადაც $C$ ნებისმიერი მუდმივაა. მისი კონკრეტული მნიშვნელობა დამოკიდებულია ნულოვანი დონის არჩევაზე. ჩვეულებრივ, ნულად თვლიან
უსასრულოდ დაშორებული სხეულების პოტენციალურ ენერგიას. ამ შემთხვევაში	$C = 0$ და $$U = -G\frac{m_1 m_2}{r}$$.
	
	\section{კოსმოსური სიჩქარეები}
	\subsection{პირველი კოსმოსური სიჩქარე}
		პირველი კოსმოსური სიჩქარე არის ის სიჩქარე, რომელიც საჭიროა სხეულს მივანიჭოთ გასროლისას რომ არ დაეცეს დედამიწაზე და გააგრძელოს მის გარშემო ბრუნვა. ნიუტონის მეორე კანონით:
		\begin{equation}
			\frac{mv^2}{r_E} = G\frac{M_Em}{r_E^2}
		\end{equation}
სადაც $M_E$ არის დედამიწის მასა, $r_E$ არის დედამიწის რადიუსი. რიცხვით გამოთვლისას მიიღება რომ $v = 7.91 \cdot 10^3$ მ/წმ.

	\subsection{მეორე კოსმოსური სიჩქარე}
		მეორე კოსმოსური სიჩქარის მინიჭებისას სხეულს შეუძლია დატოვოს დედამიწის ორბიტა, თუკი ჩავწერთ სრულ მექანიკურ ენერგიას. 
			\begin{equation}
				E = \frac{mv^2}{2} - G\frac{M_E m}{r_E}
			\end{equation}
ცხადოა როდესაც დედამიწის დატოვებს მას აღარ ექნება დედამიწასთან ურთიერთქმედების პოტენციალური ენერგია, და რადგან მინიმალურ სიჩქარეს ვეძებთ აღარც კინეტიკური ენერგია ექნება ორბიტის დატოვებისას მაშინ.
			\begin{equation}
				\frac{mv^2}{2} - G\frac{M_E m}{r_E} = 0
			\end{equation}
აქედან მივიღებთ:
			\begin{equation}
				v = \sqrt{\frac{2GM_E}{r_E}}
			\end{equation}
სადაც $M_E$ არის დედამიწის მასა, $r_E$ არის დედამიწის რადიუსი. რიცხვით გამოთვლისას მიიღება რომ $v = 11.2 \cdot 10^3$ მ/წმ.

ორივე შემთხვევაში შეიძლება გამოვიყენოთ მიახლოება, $g = GM/r_E^2$ და ზემო განტოლებებში ჩავსვათ.


\section{ამოცანები}
\textbf{01.} რა დროში დაეცემა მთვარე დედამიწას თუ ის სწრაფად გაჩერდება.\\
ამოხნსა: ამ ამოცანაში უნდა გამოვიყენოთ კეპლერის მესამე კანონი:
	\begin{equation}
		\frac{T_1^2}{T_2^2} = \frac{a_1^3}{a_2^3}
	\end{equation}
დავარდნა შეიძლება განვიხილოთ როგორც ძალიან გაწელილი ელიფსი. თუ დავუშვებთ რომ თავიდან მთვარის რადიუსი იყო $a$ ახალი რადიუსი იქნება $a/2$, მაშინ ვარდნის დრო იქნება.
	\begin{equation}
		T_1^2 = T_2^2\cdot\frac{(a/2)^3}{a^3} = T_2^2 \frac{1}{8}
	\end{equation}
სადაც $T_2$ არის ძველი მთვარის პერიოდი, მაშინ დავარდნის დრო იქნება პერიოდის ნახევარი $T_1/2$

\textbf{02.} უძრავად დამაგრებული $M$ მასის ნივთიერი წერტილის გრავიტაციულ ველში დიდი მანძილით დაშორებული წერტილიდან (ამ მანძილზე
გრავიტაციული ურთიერთქმედება შეგვიძლია უგულებელვყოთ) $v$ სიჩქარით მოძრაობს $m$ მასის ნივთიერი წერტილი, რომლის სამიზნე პარამეტრია $\rho$. იპოვეთ უმცირესი მანძილი ნივთიერ წერტილებს შორის.
		\begin{figure}[H]
           \includegraphics[width=\linewidth = 0.6]{figures/fig_1}
           \caption{A boat.}
           \label{fig:boat1}
        \end{figure}
        
ამოხსნა:	იხსნება იმპულსის მუდმივობისა და ენერგიის მუდმივობით.\\
პასუხი: $$ r_{min} = \frac{1}{v^2} $$
\end{document}