
\documentclass[a4paper]{article}

%% Language and font encodings
\usepackage[T1,T8K,T8M]{fontenc}
\usepackage[utf8]{inputenc}
\usepackage[english,georgian]{babel}

%% Sets page size and margins
\usepackage[a4paper,top=3cm,bottom=2cm,left=3cm,right=3cm,marginparwidth=1.75cm]{geometry}

%% Useful packages
\usepackage{amsmath}
\usepackage{graphicx}
\usepackage[colorinlistoftodos]{todonotes}
\usepackage[colorlinks=true, allcolors=blue]{hyperref}
\usepackage{float}
\usepackage{enumerate}
\usepackage{subfig}

\title{კეპლერის კანონები}
\author{ლევან კანკაძე}

\begin{document}
	\maketitle
	
	\begin{abstract}
		კეპლერის კანონები.
	\end{abstract}
	
	\section{სტატიკა} სტატიკაში შეისწავლება მყარი სხეულების წონასწორობა, რომელზეც მოქმედებს ძალები. წონასწორობაში იგულისხმება მდგომარეობა, რომლისთვისაც, სხეულს არ გააჩნია აჩქარება, ანუ მოძრაობს თანაბრად და წრფივად, ან ნაწილობრივ, იმყოფება უძრავად ათვლის ინერციულ სისტემაში. (პრაქტიკულად ამოცანებში, დედამიწასთან დაკავშირებული ათვლის სისტემა ითვლება ინერციულად.
	
	რა ძალები მოქმედებს წონასწორობაში მყოფ სხეულზე? პირველ რიგში უნდა გავიხსენოთ სიმძიმის ძალა. ეს სიმძიმის ძალა არის ტოლქმედი სხეულის შემადგენელი ნაწილაკების სიმძიმის ძალისა.
	
	შემდეგ მოქმედებს ბმის რეაქციის ძალები - ეს ძალები ეწინააღმდეგება სხეულის მოძრაობას რომელიმე მიმართულებით.
	
	
	ამიტომაც განვიხილოთ როგორაა მიმართული რამდენიმე სახის ბმის რეაქციის ძალები:
	
	1. 
	
	2. გადაბმა არის დრეკადი ძაფით, მაშინ დრეკადობის ძალა არის ყოველთვის მიმართული ძაფის გასწვრივ და "გამოდის" იმ წერტილიდან რომლითაც მიმაგრებულია სხეულზე.
		\begin{figure}[H]
	 	\centering
           \includegraphics[width=0.2\columnwidth]{figures/static_c}
           \caption{A boat.}
           \label{fig:boat1}
        \end{figure}
	 
	 3. სახსრული შეერთება - 
	 
	\section{შენახვის კანონები დაჯახებებში}
	 
	\section{სითბო}	
	თუ ნივთიერება დნება $+\lambda m$ გამყარება $-\lambda m$\\
	თუ ნივთიერება ორთქლდება $+r m$ კონდესირდება $-rm$
	 
	\section{შენახვის კანონები}
	\textbf{01.} $m$ მასის უძრავ ბირთვს $V$ სიჩქარით ეჯახება $M$ მასის მოძრავი ბირთვი. იპოვეთ ბირთვების სიჩქარეები დაჯახების შემდეგ, თუ დაჯახება დრეკადია და ცენტრული. ძალის მოქმედებს წრფე გადის სხეულის მასათა ცენტრზე - სიმძიმის ცენტრი.
		
	\section{ჭოჭონაქები}
	\textbf{01.} იპოვეთ რა ძალით მოქმედებს ჭერზე, ნახატზე გამოსახული უმასო ჭოჭონაქების სისტემა. თოკები უჭიმვადია და უმასო,  თითოეული სხეულის მასაა $m$.  ხახუნი უგულებელყავით.
	 			\begin{figure}[H]
	 			\centering
           \includegraphics[width=0.2\columnwidth]{figures/03}
           \caption{A boat.}
           \label{fig:boat1}
        \end{figure}
	 
	\section{წრეწირზე მოძრაობა}
	 \textbf{01.} მოტოციკლეტისტი მოძრაობს ჰორიზონტალურ ზედაპირზე $v = 70$ კმ/სთ სიჩქარით, ბრუნდება $R = 100$ მ რადიუსის მოსახვევში, რა კუთხით უნდა გადაიხაროს რომ არ დაეცეს? \\
	 ამოხსნა\\
	 აქაც ხახუნის ძალაა, ძალა რომელიც აჩერებს მოტოციკლისტს, $F_{fr} = \frac{m v^2}{R}$, საყრდენის რეაქციის ძალა $N = mg$. მომენტების წესი სიმძიმის ცენტრის მიმართ მომცემს განტოლებას $F_{fr}\cdot l \sin \alpha = N l \cos \alpha$. აქ მოცემული არაა $\mu$ და მაგიტომ გვჭირდება. ეს მომენტები.
	 
	\textbf{02.} რა მაქსიმალური $v$ სიჩქარით შეიძლება იმოძრაოს მანქანამ $\alpha$ კუთხით დახრილ სიბრტყეზე თუ სიმრუდის რადიუსია $R$ და ხახუნის კოეფიციენტი ბორბლებსა და გზას შორის არის $k$.
			\begin{figure}[H]
           \includegraphics[width=0.2\columnwidth]{figures/02}
           \caption{A boat.}
           \label{fig:boat1}
        \end{figure}
	 
	\section{კეპლერის კანონები}
	\subsection{კეპლერის პირველი კანონი}
	პლანეტები მოძრაობს ელიფსებზე, რომელთა ერთ-ერთ ფოკუსში იმყოფება მზე.
	\subsection{კეპლერის მეორე კანონი}
	პლანეტის რადიუს-ვექტორი დროის ტოლ შუალედებში ტოლ ფართობებს მოხვეტს.
	\subsection{კეპლერის მესამე კანონი}
	პლანეტების გარშემოვლის პერიოდების კვადრატები ისე შეეფარდება ერთმანეთს, როგორც მათი ორბიტების დიდი ნახევარღერძების კუბები.
	\begin{equation}
		\frac{T_1^2}{T_2^2} = \frac{a_1^3}{a_2^3}
	\end{equation}
	\section{გრავიტაციული ურთიერთქმედების პოტენციალური ენერგია}
	$r$ მანძილით დაშორებული $m_1$ და $m_2$ მასის ნივთიერი წერტილების გრავიტაციული ურთიერთქმედების პოტენციალური ენერგიის ფორმულის მიღებას ინტეგრების ცოდნა სჭირდება. ჩვენ მოვიყვანთ შედეგს გამოყვანის გარეშე:
	\begin{equation}
		U = -G\frac{m_1 m_2}{r} + C
	\end{equation}
სადაც $C$ ნებისმიერი მუდმივაა. მისი კონკრეტული მნიშვნელობა დამოკიდებულია ნულოვანი დონის არჩევაზე. ჩვეულებრივ, ნულად თვლიან
უსასრულოდ დაშორებული სხეულების პოტენციალურ ენერგიას. ამ შემთხვევაში	$C = 0$ და $$U = -G\frac{m_1 m_2}{r}$$.
	
	\section{კოსმოსური სიჩქარეები}
	\subsection{პირველი კოსმოსური სიჩქარე}
		პირველი კოსმოსური სიჩქარე არის ის სიჩქარე, რომელიც საჭიროა სხეულს მივანიჭოთ გასროლისას რომ არ დაეცეს დედამიწაზე და გააგრძელოს მის გარშემო ბრუნვა. ნიუტონის მეორე კანონით:
		\begin{equation}
			\frac{mv^2}{r_E} = G\frac{M_Em}{r_E^2}
		\end{equation}
სადაც $M_E$ არის დედამიწის მასა, $r_E$ არის დედამიწის რადიუსი. რიცხვით გამოთვლისას მიიღება რომ $v = 7.91 \cdot 10^3$ მ/წმ.

	\subsection{მეორე კოსმოსური სიჩქარე}
		მეორე კოსმოსური სიჩქარის მინიჭებისას სხეულს შეუძლია დატოვოს დედამიწის ორბიტა, თუკი ჩავწერთ სრულ მექანიკურ ენერგიას. 
			\begin{equation}
				E = \frac{mv^2}{2} - G\frac{M_E m}{r_E}
			\end{equation}
ცხადოა როდესაც დედამიწის დატოვებს მას აღარ ექნება დედამიწასთან ურთიერთქმედების პოტენციალური ენერგია, და რადგან მინიმალურ სიჩქარეს ვეძებთ აღარც კინეტიკური ენერგია ექნება ორბიტის დატოვებისას მაშინ.
			\begin{equation}
				\frac{mv^2}{2} - G\frac{M_E m}{r_E} = 0
			\end{equation}
აქედან მივიღებთ:
			\begin{equation}
				v = \sqrt{\frac{2GM_E}{r_E}}
			\end{equation}
სადაც $M_E$ არის დედამიწის მასა, $r_E$ არის დედამიწის რადიუსი. რიცხვით გამოთვლისას მიიღება რომ $v = 11.2 \cdot 10^3$ მ/წმ.

ორივე შემთხვევაში შეიძლება გამოვიყენოთ მიახლოება, $g = GM/r_E^2$ და ზემო განტოლებებში ჩავსვათ.

\section{გეომეტრიული ოპტიკა} თუ სხეულს დავანათებთ წერტილოვანი წყაროდან, მაშინ საგნის ჩრდილი იქნება სრული, მკვეთრად შემოხაზული საზღვარით. 
%https://en.wikipedia.org/wiki/Umbra,_penumbra_and_antumbra#Penumbra
			\begin{figure}[H]
           \includegraphics[width=0.2\columnwidth]{figures/shadow}
           \caption{A boat.}
           \label{fig:boat1}
        \end{figure}
ნახევარჩრდილის ზომის და გეომეტრიული ფორმის განსაზღვრა შესაძლებელია გეომეტრიული აგებით, სინათლის წრფივი გავრცელების მიხედვით.

თუკი ობიექტს ვანათებთ გაწელილი არაწერტილოვანი სინათლის წყაროთი, მაშინ ის ასევე წარმოქმნის ნახევარჩრდილს-ნაწილობრივ განათებულ ეკრანის არეს, სადაც მხოლო მანათობელი ობიექტის ნაწილიდან ეცემა სინათლე. ზოგიერთ შემთხვევაში შეიძლება სრული ჩრდილი საერთოდ არ გვქონდეს, და მხოლოდ იყოს ნახევარჩრდილი.
			\begin{figure}[H]
           \includegraphics[width=0.2\columnwidth]{figures/penumbra}
           \caption{A boat.}
           \label{fig:boat1}
        \end{figure}
  	 

\section{ამოცანები}
\textbf{01.} რა დროში დაეცემა მთვარე დედამიწას თუ ის სწრაფად გაჩერდება.\\
ამოხნსა: ამ ამოცანაში უნდა გამოვიყენოთ კეპლერის მესამე კანონი:
	\begin{equation}
		\frac{T_1^2}{T_2^2} = \frac{a_1^3}{a_2^3}
	\end{equation}
დავარდნა შეიძლება განვიხილოთ როგორც ძალიან გაწელილი ელიფსი. თუ დავუშვებთ რომ თავიდან მთვარის რადიუსი იყო $a$ ახალი რადიუსი იქნება $a/2$, მაშინ ვარდნის დრო იქნება.
	\begin{equation}
		T_1^2 = T_2^2\cdot\frac{(a/2)^3}{a^3} = T_2^2 \frac{1}{8}
	\end{equation}
სადაც $T_2$ არის ძველი მთვარის პერიოდი, მაშინ დავარდნის დრო იქნება პერიოდის ნახევარი $T_1/2$

\textbf{02.} უძრავად დამაგრებული $M$ მასის ნივთიერი წერტილის გრავიტაციულ ველში დიდი მანძილით დაშორებული წერტილიდან (ამ მანძილზე
გრავიტაციული ურთიერთქმედება შეგვიძლია უგულებელვყოთ) $v$ სიჩქარით მოძრაობს $m$ მასის ნივთიერი წერტილი, რომლის სამიზნე პარამეტრია $\rho$. იპოვეთ უმცირესი მანძილი ნივთიერ წერტილებს შორის.
		\begin{figure}[h]
           \includegraphics[width=0.2\columnwidth]{figures/fig_1}
           \caption{A boat.}
           \label{fig:boat1}
        \end{figure}
        
ამოხსნა:	იხსნება იმპულსის მუდმივობისა და ენერგიის მუდმივობით.\\
პასუხი: $$ r_{min} = \frac{1}{v^2} $$

\section{გამოსახულების აგება ლინზებსა და სფერულ სარკეებში}
ლინზით ან სარკით მიღებული გამოსახულების ადგილმდებარეობის განსაზღვრა შეიძლება ორი მეთოდით - ალგებრული გამოთვლით (ლინზისა და სარკის ფორმულის გამოყენებით) ანდა გეომეტრიული აგებით.

პირველი მეთოდი თუმც არის უფრო უნივერსალური, ხშირად რთულ ოპტიკურ სისტემებში მას თავს ვერ ავარიდებთ. სამაგიეროდ მეორე მეთოდი უფრო თვალსაჩინოა. ამიტომაც ალგებრულად ამოცანის შემთხვევაშიც კი ვაკეთებთ ნახაზს, რომელიც გვეხმარება საჭირო სისტემის დაწერაში. თუ ამოცანა არ არის ზედმეტად შრომატევადი(?), აგებით ამოხსნა არის უფრო მოსახერხებელი.

თხელ ლინზებში გამოსახულების აგებისას ვსარგებლობთ სამი ძირითადი თვისებით სინათლის სხივის ნახ.ა)~\ref{fig:optics_1}.
		\begin{figure}[h]
		   \centering
           \includegraphics[width=0.5\columnwidth]{figures/optics_1}
           \caption{სხივთა სვლა თხელ ა) შემკრებ, ბ) გამბნევ ლინზაში.}
           \label{fig:optics_1}
        \end{figure}

1) სხივი $AA_1$, რომელიც გადის ლინზის ოპტიკურ ცენტრში $O$ (მეორენაირად ეძახიან დამხმარე ოპტიკურ ღერძს) არ გარდატყდება.

2) სხივი $BB_1$,რომელიც ეცემა ლინზას მთავარი ოპტიკური ღერძის პარალელურად გარდატყდება და გაივლის ლინზის უკანა $F'$ ფოკუსსი.

3) სხივი $CC_1$, რომელიც გადის წინა ფოკუსში $F$, ლინზაში გარდატეხის მერე გამოდის მთავარი ოპტიკური ღერძის პარალელურად.

უკანა ფოკუსი $F'$ ეწოდება წერტილს რომელშიც იკრიბებიან გარდატეხის შემდგომ ოპტიკური ღერძის პარალელურად,ლინზაზე დაცემული სხივები. წინა $F$ და უკანა $F'$ ფოკუსები განლაგებულები არიან თხელი ლინზის მიმართ სიმეტრიულად. $F$ გადის უკანა ფოკალური სიბრტზე, $F'$-ში გადის უკანა ფოკალური სიბრტყე.

ხანდახან ასევე გვეხმარება შემდეგი წესებიც:
1) სხივები, რომლებიც ლინზას ეცემიან პარალელურ ნაკადად, გარდატეხის შემდეგ იკრიბებიან უკანა ფოკალურ სიბრტყეში~\ref{fig:optics_2}.

2) სხივები რომლებიც გამოდიან ლინზიდან პარალელურ ნაკადად, ლინზაზე დაცემამდნენ გადაიკვეთნენ წინა ფოკალურ სიბრტყეში~\ref{fig:optics_3}. 

		\begin{figure}[h]
		   \centering
           \includegraphics[width=0.5\columnwidth]{figures/optics_2}
           \caption{ლინზაზე დაცემულ პარალელურ სხივთა სვლა თხელ ა) შემკრებ, ბ) გამბნევ ლინზაში.}
           \label{fig:optics_2}
        \end{figure}

		\begin{figure}[h]
		   \centering
           \includegraphics[width=0.5\columnwidth]{figures/optics_3}
           \caption{ლინზიდან გამოსული პარალელურ სხივთა "უკუსვლა" თხელ ა) შემკრებ, ბ) გამბნევ ლინზაში.}
           \label{fig:optics_3}
        \end{figure}
        
%\begin{figure}
%\centering
%\begin{minipage}{.5\textwidth}
%  \centering
%  \includegraphics[width=.9\linewidth]{figures/optics_2}
%  \captionof{figure}{სხივთა სვლა თხელ ა) შემკრებ, ბ) გამბნევ ლინზაში.}
%  \label{fig:optics_1}
%\end{minipage}%
%\begin{minipage}{.5\textwidth}
%  \centering
%  \includegraphics[width=.9\linewidth]{figures/optics_2}
%  \captionof{figure}{ლინზაზე დაცემულ პარალელურ სხივთა სვლა თხელ ა) შემკრებ, ბ) გამბნევ ლინზაში.}
%  \label{fig:optics_2}
%\end{minipage}
%\end{figure}
        
\end{document}