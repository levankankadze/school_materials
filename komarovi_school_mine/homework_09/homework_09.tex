%This is my super simple Real Analysis Homework template

\documentclass{article}

%% Language and font encodings
\usepackage[T1,T8K,T8M]{fontenc}
\usepackage[utf8]{inputenc}
\usepackage[english,georgian]{babel}

\usepackage{amsmath}
\usepackage{graphicx}
\usepackage[colorinlistoftodos]{todonotes}
\usepackage[colorlinks=true, allcolors=blue]{hyperref}
\usepackage{float}
\usepackage{enumerate}
\usepackage{subfig}
\usepackage{gensymb}

%\title{დავალება 01}
%\author{Your Name}
%\date\today
%This information doesn't actually show up on your document unless you use the maketitle command below

\begin{document}

\subsection{ამოცანა 1}
განსაზღვრე ბენზინის ხარჯი მანქანის $S = 1$ კმ-ის გავლისას, $v = 60$ კმ/სთ სიჩქარით მოძრაობის დროს. ძრავის სიმძლავრეა $N = 23$ ცხენის ძალა. ძრავის მარგი ქმედების კოეფიციენტი $\eta = 30 ~ \%$, ბენზინის წვის კუთრი სითბო $q = 45 \cdot 10^6$ ჯ/კგ. 1 ცხენის ძალა არის 746 ჯოული.

\subsection{ამოცანა 2}
როგორი უნდა იყოს იმ ტუმბოს მინიმალურ სიმძლავრე, რომელსაც $S = 0.05~\text{სმ}^2$ განივკვეთის მილით $V = 0.2 ~ \text{მ}^3/text{წმ}$ წყალი $h = 5$ მ სიმაღლეზე ააქვს? რა შეიცვლება, თუ საჭირო გახდება ასეთივე მასის ცემენტის ხსნარის ატანა? ხსნარის სიმკვრივე ორჯერ აღემატება წყლის სიმკვრივეს.


\end{document}